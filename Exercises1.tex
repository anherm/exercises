\documentclass[11pt, a4paper, twoside]{article}
\usepackage{geometry}
\geometry{a4paper,left=35mm,right=25mm, top=35 mm, bottom=40 mm}
\setlength{\parindent}{0pt}
\usepackage[english]{ babel }
\usepackage[latin1]{inputenc}
\usepackage[T1]{fontenc}
\usepackage{amsmath}
\usepackage{amsthm}
\usepackage{amssymb}
\usepackage{nicefrac}
\usepackage{mathrsfs}
\usepackage[arrow, matrix, curve]{xy}
\usepackage{enumitem}
\usepackage{mdwlist}
\usepackage{makeidx}
\usepackage{fancyhdr}
\usepackage{tikz}
\usepackage{algorithm2e}
\usepackage{bbm}

\begin{document}
 
\theoremstyle{definition}
\newtheorem{Def} {Definition} 
\newtheorem{Bsp} [Def] {Beispiel}
\theoremstyle{definition}
\newtheorem{Bem} [Def] {Bemerkung}
\theoremstyle{plain}
\newtheorem{Ex} [Def] {Exercise}
\newtheorem{Thm} [Def] {Theorem}
\newtheorem*{Thm*}{Theorem}
\newtheorem{Kor} [Def] {Korollar}
\bibliographystyle{alphadin}

\fancyhead{}
\renewcommand{\headrulewidth}{0pt}
\newpage
\fancyhead{}                            % Zun�chst ist alles leer.
\fancyhead[CE]{\leftmark}               % Links bei geraden und rechts
%bei ungeraden Seitenzahlen soll der Name der Section stehen.
\fancyhead[CO]{\rightmark}
\renewcommand{\headrulewidth}{0.4pt}
\newcommand{\Z}{\mathbb Z}
\newcommand{\R}{\mathbb R}
\newcommand{\PE}{\mathcal P}
\section*{Lattices and Convex Bodies: Exercises 1}

\begin{Ex}

Let $B \in \R^{d \times k}$ be a lattice basis and $U \in \R^{k \times k}$. Then $BU$ is a basis of $\Lambda(B)$ if and only if $U$ is unimodular.

\begin{proof}

First assume that $BU$ is a basis of $\Lambda(B)$. Thus we can write $B = BUV$ for some $V \in \Z^{k \times k}$. Since $B$ is linear independent and therefore has a left inverse, we can conclude $\mathbbm 1 = UV$. We obtain $1 = det(U) det(V)$ and thus $det(U) \in \{\pm 1\}$. 

For the second implication, assume that $U$ is unimodular. We want to show that $\Lambda(B) = \Lambda(BU)$. For the inclusion from the right to the left, let $x \in BU$ be a basis vector and write $x = Bu$ for some column $u \in U$. Since $U$ is unimodular, we especially have $u \in \Z^k$, and thus $x \in \Lambda(B)$. Using this implication and the fact that $U$ is invertible, we can now write

\[\Lambda(B) = \Lambda(B(UU^{-1})) = \Lambda((BU)U^{-1})) \subset \Lambda(BU) \]

since $U^{-1}$ is also unimodular. 
\end{proof}

\end{Ex}

\begin{Ex}

Let $\Lambda' \subset \Lambda$ be full-dimensional lattices, $B'$ a basis of $\Lambda'$. Then we have

\[ |(\PE_{B'} \cap \Lambda)| = \frac{\text{det}(\Lambda')}{\text{det}(\Lambda)} = [\Lambda \colon \Lambda']\,.\]


\begin{proof}

We already know that $\text{det}(\Lambda) = \text{vol}(\PE_{B})$ and $\text{det}(\Lambda') = \text{vol}(\PE_{B'})$. Thus the quotient $\frac{\text{det}(\Lambda')}{\text{det}(\Lambda)} = \frac{\text{vol}(\PE_{B'})}{\text{vol}(\PE_{B})}$ is the number of parallelipipeds $\PE_{B}$ that fit into $\PE_{B'}$. But each parallelepiped can be identified with its corner in the corresponding lattice that has the lowest coordinate in every dimension. Thus each point in $\PE_{B'} \cap \Lambda$ defines exactly one of the $\PE_{B}$ inside $\PE_{B'}$, and this enumerates all of them. Hence we have the first equality  $\frac{\text{det}(\Lambda')}{\text{det}(\Lambda)} = |(\PE_{B'} \cap \Lambda)|$. 

For the equality $|(\PE_{B'} \cap \Lambda)| = [\Lambda \colon \Lambda']$ note that $\PE_{B'} \cap \Lambda$ is a system of representatives for the equivalence classes of the quotient group $\Lambda / \Lambda'$.

\end{proof}

\end{Ex}

\begin{Ex}

\end{Ex}

\begin{Ex}

\begin{enumerate}

\item Let $\Lambda \subset \R^d$ be a subspace spanned by lattice vectors. Then $\Lambda \cap U$ is a lattice with $\text{dim}(\Lambda \cap U) = \text{dim} (U)$.

\item Let $p \colon \R^d \to U^\bot$ be the orthogonal projection onto the orthogonal complement of $U$. Then $p(\Lambda)$ is a lattice with $\text{dim}(\Lambda) = \text{dim}(\Lambda) - \text{dim}(U)$.
\end{enumerate}

\begin{proof}

1. Since both a vector space and a lattice are closed under addition, $\Lambda \cap U$ is an additive subspace of $\R^d$. Since $\Lambda \cap U \subset \Lambda$, the discreteness of $\Lambda \cap U$ is inherited from $\Lambda$. Thus $\Lambda \cap U$ is a lattice.

For the statement about the dimension of $\Lambda \cap U$, let $U = \{x_1, \dotsc, x_m\}$, where $x_i \in \Lambda$ are the lattice vectors that span $U$. Remove some of the $x_i$ until the remaining set of $x_i$ is linearly independant and thus a basis of $U$. Now we see that $\text{dim}(U) \leq \text{dim}(\Lambda \cap U)$ since each of the basis vectors $x_i$ of $U$ is also in $\Lambda \cap U$. The other inequation $\text{dim}(\Lambda \cap U) \leq \text{dim}(U)$ is clear because $\Lambda \cap U \subset U$.

2. VAGUELY ABOUT $\bigoplus$.

Let $B'$ be a basis of the $\text{dim}(U)$-dimensional lattice $\Lambda' := \Lambda \cap U$. Extend $B'$ to a basis $B$ of $\Lambda$. Now $\Lambda(p(B)) = \Lambda(p(B\backslash B'))$ since $p(B') = 0$. We see that the set $p(B\backslash B')$ is linearly independant, since if there are $x_1 = p(y_1), \dotsc x_k = p(y_k) \in p(B\backslash B')$ with 

\[ 0 = \sum_{i = 1}^{k} \lambda_i x_i = p( \sum_{i = 1}^k \lambda_i y_i) \]

for some $\lambda_i \in \R$, we know that the right hand-side is not in the kernel of $p$ unless all $\lambda_i = 0$. Now it suffices to show that $p(\Lambda) = \Lambda(p(B \backslash B'))$, which is quite clear. This shows both that $p(\Lambda)$ is a lattice and that $\text{dim}(\Lambda) = \text{dim}(\Lambda) - \text{dim}(U)$.

\end{proof}

\end{Ex}

\begin{Ex}

\begin{enumerate}

\item The size reduction step of the LLL algorithm can be performed using $\text{poly}(d)$ arithmetic operations.

\item

\end{enumerate}

\begin{proof}

For the proof, we give a concrete algorithm for the size reduction and proof its correctness.

\begin{center}
\begin{algorithm}[H]
\caption{Size Reduction}
\DontPrintSemicolon
\SetKwInOut{Input}{Input}
\SetKwInOut{Output}{Output}

\Input{A basis $B$ of a lattice $\Lambda$}
\Output{A basis $B$ of $\Lambda$ s.\,t. $|\mu_{ji}| \leq \frac{1}{2}$ for all $j < i$}  
\For{$i = 2$ \KwTo $n$}  
{
	\For{$j = i - 1$ \KwTo $1$}
	{
		$b_{i} = b_i - \left\lceil\mu_{ji}\right\rfloor b_j$\;
	}
}
\end{algorithm}
\end{center}

\begin{Thm*}

The above algorithm performs the Size Reduction correctly, i.\,e. after performing it, we have 
\[|\mu_{ji}| \leq \frac{1}{2} \text{ for all } j < i\,,\]
and the resulting basis is still a basis of the input lattice $\Lambda$.
The algorithm can be performed in time poly(d).

\begin{proof}
The statement about the running time is clear. The correctness is shown in the following three claims.

\emph{Claim 1: Let $B$ be the input basis, $B'$ the output basis. Then we have $\Lambda(B) = \Lambda(B')$.}

In each step of the algorithm, the only change applied to the basis is to exchange some basis vector $b_i$ with $b_i' := b_i - \left\lceil\mu_{ji}\right\rfloor b_j$. But since $\left\lceil\mu_{ji}\right\rfloor \in \Z$ and since we also can write $b_i = b_i' + \left\lceil\mu_{ji}\right\rfloor b_j$, this does not change the lattice $\Lambda$. 

\emph{Claim 2: Consider the iteration of the algorithm where we set $b'_{i} = b_i - \left\lceil\mu_{ji}\right\rfloor b_j$ with $j < i$. Then we have $|\mu'_{ji}| \leq \frac{1}{2}$.} 

As a preparation, we compute

\[b_j^* b_j = b_j^*\left(b_j^* + \sum_{h < j} \mu_{hj} b_h^*\right) = b_j^* b_j^*\]

since $b_j^*$ is orthogonal to all $b_h^*$ with $h < j$. Thus we can conclude

\begin{align*}
 |\mu'_{ji}| &= \left|\frac{b_j^* b'_i}{b_j^*b_j^*}\right| \\
             &= \left|\frac{b_j^* \left(b_i - \left\lceil\mu_{ji}\right\rfloor b_j\right)}{b_j^*b_j^*}\right| \\
             &= \left|\frac{b_j^* b_i}{b_j^*b_j^*} - \left\lceil\mu_{ji}\right\rfloor \frac{b_j^*b_j}{b_j^*b_j^*}\right| \\
             &= \left|\mu_{ji} - \left\lceil\mu_{ji}\right\rfloor \frac{b_j^*b_j^*}{b_j^*b_j^*}\right| \\
             &\leq \frac{1}{2}\,.
\end{align*}

\emph{Claim 3: Again consider the iteration for the pair $j < i$. For all $j' < i'$ which have been considered before, we still have $|\mu_{j'i'}| \leq \frac{1}{2}$.}

First consider the case that $i' < i$. Then we also have $j' < i' < i$, so neither $b_{j'}^{(*)}$ nor $b_{i'}^{(*)}$ are affected by the change in $b_i$ and $\mu_{j'i'}$ remains the same.
The other case to study is $i' = i$ and, due to the fact that the inner loop goes downwards, $j < j' < i$. As above, we can compute

\[|\mu'_{j'i}| = \left|\mu_{j'i} - \left\lceil\mu_{ji}\right\rfloor \frac{b_{j'}^*b_j}{b_{j'}^*b_{j'}^*}\right|\,.\]

But we have

\[ b_{j'}^*b_j = b_{j'}^* \left(b_j^* + \sum_{h < j} \mu_{hj} b_h^*\right) = b_{j'}^*b_j^* + \sum_{h < j} \mu_{hj} b_{j'}^*b_h^* = 0\, , \]

where $b_{j'}^*$ is orthogonal to $b_j^*$ and to all $b_h^*$ for $h < j$ since $h < j < j'$. Note that this is the step where we need the inner loop to go downwards. However, now the right summand in the above equation vanishes and we get

\[|\mu'_{j'i}| = \left|\mu_{j'i} \right| \leq \frac{1}{2} \]

since we already have considered the iteration $j' < i$ as $j' > j$ and the inner loop goes downwards.
 
Thus we have shown that the basis we produce by the size reduction algorithm is a basis of $\Lambda$, and that in each iteration we assure for one pair $j < i$ that $\left |\mu_{ji}\right| \leq \frac{1}{2}$ without destroying this property for any other pair $j' < i'$ considered before, and since we consider all pairs $j < i$, the proof is complete.
 
\end{proof}

\end{Thm*}

\end{proof}

\end{Ex}

\end{document}