\documentclass[11pt, a4paper, twoside]{article}
\usepackage{geometry}
\geometry{a4paper,left=35mm,right=25mm, top=35 mm, bottom=40 mm}
\setlength{\parindent}{0pt}
\usepackage[english]{ babel }
\usepackage[latin1]{inputenc}
\usepackage[T1]{fontenc}
\usepackage{amsmath}
\usepackage{amsthm}
\usepackage{amssymb}
\usepackage{nicefrac}
\usepackage{mathrsfs}
\usepackage[arrow, matrix, curve]{xy}
\usepackage{enumitem}
\usepackage{mdwlist}
\usepackage{makeidx}
\usepackage{fancyhdr}
\usepackage{tikz}
\usepackage{algorithm2e}

\begin{document}
 
\theoremstyle{definition}
\newtheorem{Def} {Definition} 
\newtheorem{Bsp} [Def] {Beispiel}
\theoremstyle{definition}
\newtheorem{Bem} [Def] {Bemerkung}
\theoremstyle{plain}
\newtheorem{Ex} [Def] {Exercise}
\newtheorem{Thm} [Def] {Theorem}
\newtheorem{Kor} [Def] {Korollar}
\bibliographystyle{alphadin}

\fancyhead{}
\renewcommand{\headrulewidth}{0pt}
\newpage
\fancyhead{}                            % Zun�chst ist alles leer.
\fancyhead[CE]{\leftmark}               % Links bei geraden und rechts
%bei ungeraden Seitenzahlen soll der Name der Section stehen.
\fancyhead[CO]{\rightmark}
\renewcommand{\headrulewidth}{0.4pt}
\newcommand{\Z}{\mathbb Z}
\newcommand{\R}{\mathbb R}
\section*{Lattices and Convex Bodies: Exercises 2}

\begin{Ex}

\begin{enumerate}

\item Show that $(\frac{1}{2} \Lambda) / \Lambda \cong (\Z_2)^d$.
\item Show that $|(\Z_2)^d| = 2^d = \frac{\text{det}\Lambda}{\text{det}\frac{1}{2}\Lambda}$.

\end{enumerate}

\begin{proof}

1. Let $B$ be some basis of $\Lambda$. Note that $B' := \frac{1}{2} B$ is a basis of $\frac{1}{2} \Lambda$. Define a map $(\frac{1}{2} \Lambda) / \Lambda \to (\Z_2)^d$ by mapping an element $x = \sum_{i = 1}^d \lambda_i b_i'$ with $b_i'$ the columns of $B'$ and $\lambda_i \in \Z$ to $(\lambda_i \text{ mod }2)_{i = 1 \dotsb d}$. This map is well-defined since it maps $\Lambda$ to $0$. An inverse is given by sending a tuple $(\lambda_i)_{i = 1 \dotsc d} \in (\Z_2)^d$ to $\sum_{i = 1}^d \lambda_i b_i'$.

2. Since an element of $(\Z_2)^d$ consists of $d$ variables that can independantly be chosen to be either $0$ or $1$, we have $|(\Z_2)^d| = 2^d$. By the first statement and exercise 1.1, we get $|(\Z_2)^d| = |(\frac{1}{2} \Lambda) / \Lambda| = |(\frac{1}{2} \Lambda) : \Lambda| = \frac{\text{det}\Lambda}{\text{det}\frac{1}{2}\Lambda}$.

\end{proof}

\end{Ex}

\end{document}