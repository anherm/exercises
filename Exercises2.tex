\documentclass[11pt, a4paper, twoside]{article}
\usepackage{geometry}
\geometry{a4paper,left=35mm,right=25mm, top=35 mm, bottom=40 mm}
\setlength{\parindent}{0pt}
\usepackage[english]{ babel }
\usepackage[latin1]{inputenc}
\usepackage[T1]{fontenc}
\usepackage{amsmath}
\usepackage{amsthm}
\usepackage{amssymb}
\usepackage{nicefrac}
\usepackage{mathrsfs}
\usepackage[arrow, matrix, curve]{xy}
\usepackage{enumitem}
\usepackage{mdwlist}
\usepackage{makeidx}
\usepackage{fancyhdr}
\usepackage{tikz}
\usepackage{algorithm2e}

\begin{document}
 
\theoremstyle{definition}
\newtheorem{Def} {Definition} 
\newtheorem{Bsp} [Def] {Beispiel}
\theoremstyle{definition}
\newtheorem{Bem} [Def] {Bemerkung}
\theoremstyle{plain}
\newtheorem{Ex} [Def] {Exercise}
\newtheorem{Thm} [Def] {Theorem}
\newtheorem{Kor} [Def] {Korollar}
\bibliographystyle{alphadin}

\fancyhead{}
\renewcommand{\headrulewidth}{0pt}
\newpage
\fancyhead{}                            % Zun�chst ist alles leer.
\fancyhead[CE]{\leftmark}               % Links bei geraden und rechts
%bei ungeraden Seitenzahlen soll der Name der Section stehen.
\fancyhead[CO]{\rightmark}
\renewcommand{\headrulewidth}{0.4pt}
\newcommand{\Z}{\mathbb Z}
\newcommand{\Q}{\mathbb Q}
\newcommand{\R}{\mathbb R}
\section*{Lattices and Convex Bodies: Exercises 2}

\begin{Ex}

\begin{enumerate}

\item Show that $(\frac{1}{2} \Lambda) / \Lambda \cong (\Z_2)^d$.
\item Show that $|(\Z_2)^d| = 2^d = \frac{\text{det}\Lambda}{\text{det}\frac{1}{2}\Lambda}$.

\end{enumerate}

\begin{proof}

1. Let $B$ be some basis of $\Lambda$. Note that $B' := \frac{1}{2} B$ is a basis of $\frac{1}{2} \Lambda$. Define a map $(\frac{1}{2} \Lambda) / \Lambda \to (\Z_2)^d$ by mapping an element $x = \sum_{i = 1}^d \lambda_i b_i'$ with $b_i'$ the columns of $B'$ and $\lambda_i \in \Z$ to $(\lambda_i \text{ mod }2)_{i = 1 \dotsb d}$. This map is well-defined since it maps $\Lambda$ to $0$. An inverse is given by sending a tuple $(\lambda_i)_{i = 1 \dotsc d} \in (\Z_2)^d$ to $\sum_{i = 1}^d \lambda_i b_i'$.

2. Since an element of $(\Z_2)^d$ consists of $d$ variables that can independantly be chosen to be either $0$ or $1$, we have $|(\Z_2)^d| = 2^d$. By the first statement and exercise 1.1, we get $|(\Z_2)^d| = |(\frac{1}{2} \Lambda) / \Lambda| = |(\frac{1}{2} \Lambda) : \Lambda| = \frac{\text{det}\Lambda}{\text{det}\frac{1}{2}\Lambda}$.

\end{proof}

\end{Ex}

\begin{Ex}

Let $B \in \Q^{d \times d}$ be an LLL-reduced basis of $\Lambda$, $t \in \R^d$.

\begin{enumerate}

\item Let $x = \sum_{i = 1}^d a_i b_i$ be a shortest vector of $\Lambda$. Then $|a_j| \leq 2^{O(d)}$ for each $j = 1, \dotsc, d$.

\item A shortest vector of $\Lambda$ can be computed in time $2^{O(d^2)}$.

\end{enumerate}

\begin{proof}

Assume we have shown the first statement. Then we know that the coefficients $a_i \in \Z$ of a shortest vector $x$ are in an intervall $[-2^{O(d)}, 2^{O(d)}]$ for each $i$. Hence we find a shortest vector if we enumerate all possible choices of coefficients. To do that, we can choose each of the $d$ coefficients independently, and for each coefficient we have $2 \cdot 2^{O(d)}$ choices. This gives a runtime of $(2^{O(d)})^d = 2^{O(d^2)}$.

\end{proof}

\end{Ex}

\begin{Ex}

\end{Ex}

\begin{Ex}

\begin{enumerate}

\item Every $2$-dimensional lattice has a basis $(b_1, b_2)$ in which $\|b_1\| = \lambda_1$ and $\|b_2\| = \lambda_2$.

\item Let $\Lambda := \{x \in \Z^d \colon x_1 \equiv \dotsb \equiv x_d (\text{mod} 2)\}$, $d \geq 5$. Then there is no basis of $\Lambda$ with shortest independent vectors.

\end{enumerate}

\begin{proof}

For the first statement, choose $(b_1, b_2)$ as a reduced basis. Then by exercise 3 from sheet 1, $\|b_1\| = \lambda_1$. Now consider the lattice hyperplanes parallel to $b_1$ that contain lattice vectors to test whether there is any lattice vector linearly independent from $b_1$ and shorter than $b_2$. The only hyper planes that intersect the circuit of radius $\|b_2\|$ around $0$ go through $0$, $b_2$ and $-b_2$. On the line parallel to $b_1$ through $0$, all vectors are linear dependent from $b_1$. On the line through $b_2$, there is no lattice vector shorter than $b_2$ since $|\mu_{12}| \leq \frac{1}{2}$. For $-b_2$, one argues similarly.

For the second statement, let $v_i := 2 \cdot e_i \in \Lambda$, $i = 1, \dotsc, d$. Then $\| v_i \| = 2$ for all $i$, and the $v_i$ are linearly independent. If we consider any other non-zero $v \in \Lambda$, its absolute value is greater than $2$: for $v$ "`congruent to 0"' this is clear, and otherwise we have $\| v \| \geq \| (1, \dotsc, 1)\| > 2$ since $d > 5$. Thus $\lambda_i = 2$ for all $i = 1, \dotsc, d$ and the only chance for the desired basis is to choose all $v_i$. But these do not form a basis since all the vectors "`congruent to 1"' are not hit.

\end{proof}

\end{Ex}
\end{document}