\documentclass[11pt, a4paper, twoside]{article}
\usepackage{geometry}
\geometry{a4paper,left=35mm,right=25mm, top=35 mm, bottom=40 mm}
\setlength{\parindent}{0pt}
\usepackage[english]{ babel }
\usepackage[latin1]{inputenc}
\usepackage[T1]{fontenc}
\usepackage{amsmath}
\usepackage{amsthm}
\usepackage{amssymb}
\usepackage{nicefrac}
\usepackage{mathrsfs}
\usepackage[arrow, matrix, curve]{xy}
\usepackage{enumitem}
\usepackage{mdwlist}
\usepackage{makeidx}
\usepackage{fancyhdr}
\usepackage{tikz}
\usepackage{algorithm2e}

\begin{document}
 
\theoremstyle{definition}
\newtheorem{Def} {Definition} 
\newtheorem{Bsp} [Def] {Beispiel}
\theoremstyle{definition}
\newtheorem{Bem} [Def] {Bemerkung}
\theoremstyle{plain}
\newtheorem{Ex} [Def] {Exercise}
\newtheorem{Thm} [Def] {Theorem}
\newtheorem{Kor} [Def] {Korollar}
\bibliographystyle{alphadin}

\fancyhead{}
\renewcommand{\headrulewidth}{0pt}
\newpage
\fancyhead{}                            % Zun�chst ist alles leer.
\fancyhead[CE]{\leftmark}               % Links bei geraden und rechts
%bei ungeraden Seitenzahlen soll der Name der Section stehen.
\fancyhead[CO]{\rightmark}
\renewcommand{\headrulewidth}{0.4pt}
\newcommand{\Z}{\mathbb Z}
\newcommand{\R}{\mathbb R}
\section*{Lattices and Convex Bodies: Exercises 3}

\begin{Ex}

Let $\Lambda \subset \R^d$ be a full-dimensional lattice, $y \in \Lambda^*\backslash 0$. Then $y$ is primitive if and only if every lattice hyperplane $\{x \in \Lambda \colon y^Tx = \alpha\}$ for $\alpha \in \Z$ contains a point of $\Lambda$.

\begin{proof}

First let $y \in \Lambda^*\backslash 0$ be primitive, i.\,e. for every $t \in (0, 1)$, we have $ty \notin \Lambda^*$. By definition of $\Lambda^*$, we know that for all $x \in \Lambda$, we have $y^Tx \in \Z$. Let

\[ k := \text{gcd} \{ \alpha \in Z \colon x^Ty = \alpha \} \,.\]

THERE MIGHT BE A PROBLEM WITH INFINITE GCDS...

Note that $k \in \Z_{\geq 1}$ and not $k = \infty$ since then for all $x \in \Lambda$, we would have $x^Ty = 0$, thus $\Lambda$ would lie on some hyperplane of $\R^d$ and not be full-dimensional. Choose $t := \frac{1}{k} \in (0, 1]$. Now let $x \in \Lambda$ and compute

\[ (ty)^Tx = t (y^Tx) = \frac{\alpha}{k} \in \Z\]

for some $\alpha \in \Z$ with $k | \alpha$. Since $y$ was assumed to be primitive, we have $t = 1$ and thus $k = 1$. But then we can choose a linear combination $1 = \sum \lambda_i \alpha_i$ with $\alpha_i \in \Z$ defining hit hyperplanes and $\lambda_i \in \Z$. If $x_i$ are vectors of $\Lambda$ with $y^T x_i = \alpha_i$, we can define $x := \sum \lambda_i x_i \in \Z$ and obtain

\[ y^T x = \sum \lambda_i y^T x = \sum \lambda_i \alpha_i = 1 \,. \]

Thus the hyperplane defined by $\alpha = 1 \in \Z$ is hit, and by additivity, every other hyperplane is also hit.

On the other hand, let $y \in \Lambda^*\backslash 0$ be a vector such that for every $\alpha \in \Z$, we have 

\[\{x \in \Lambda \colon y^Tx = \alpha\} \cap \Lambda \neq \emptyset\,.\]

Assume there is some $t \in (0, 1)$ with $ty \in \Lambda^*$. Since every hyper plane is hit, there is some $x \in \Lambda$ with $y^Tx = 1$. But then we have 

\[ (ty)^Tx = t (y^Tx) = t \notin \Z \,,\]

contradicting to $ty \in \Lambda^*$. 
\end{proof}  

\end{Ex}

\end{document}